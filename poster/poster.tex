\documentclass[final,hyperref={pdfpagelabels=true}]{beamer}

\usepackage{TUINFPST}

\usepackage[backend=bibtex,sortcites=true]{biblatex}
\usepackage{lipsum}
 
%\title[Computational Intelligence]{Interactive Computer Generated Architecture}
% if you have a long title looking squeezed on the poster, just force
% some distance:
\title[Computational Intelligence]{%
  Integration of Conjunctive Queries over \\[0.2\baselineskip]%
  Description Logics into HEX-Programs %\\[0.2\baselineskip]%
}
\author[martina.muster@alumni.tuwien.ac.at]{Martina Muster}
\institute[]{%
  Technische Universit{\"a}t Wien\\[0.25\baselineskip]
  Institut f{\"u}r Informationssysteme\\[0.25\baselineskip]
  Arbeitsbereich: Wissensbasierte Systeme\\[0.25\baselineskip]
  BetreuerIn: Ao.Univ.-Prof. Dr. Maxima Musterfrau
}
\titlegraphic{\includegraphics[height=52mm]{logo_KBS_2_CMYK}}
\date[\today]{\today}
\subject{epilog}
\keywords{my kwd1, my kwd2}

%%%%%%%%%%%%%%%%%%%%%%%%%%%%%%%%%%%%%%%%%%%%%%%%%%%%%%%%%%%%%%%%%%%%%%%%%%%%%%%%%%%%%%

% Display a grid to help align images 
%\beamertemplategridbackground[12.7mm]

% play around with the background colors
% \setbeamercolor{background canvas}{bg=yellow}

% use a background picture
% \usebackgroundtemplate{%
%   \includegraphics[width=\paperwidth]{logo_KBS_2_CMYK}
% }

% play around with block colors
\setbeamercolor{block body}{fg=black,bg=white}
\setbeamercolor{block title}{fg=TuWienBlue,bg=white}

\setbeamertemplate{block begin}{
  \begin{beamercolorbox}{block title}%
    \begin{tikzpicture}%
      \node[draw,rectangle,line width=3pt,rounded corners=0pt,inner sep=0pt]{%
        \begin{minipage}[c][2cm]{\linewidth}
          \centering\textbf{\insertblocktitle}
        \end{minipage}
      };
    \end{tikzpicture}%
  \end{beamercolorbox}
  \vspace*{1cm}
  \begin{beamercolorbox}{block body}%
}

\setbeamertemplate{block end}{
  \end{beamercolorbox}
  \vspace{2cm}
}

% setup postit
\setbeamercolor{postit}{fg=black,bg=yellow} 
\newenvironment{postit}
{\begin{beamercolorbox}[sep=1em,wd=7cm]{postit}}
{\end{beamercolorbox}}


% for crop marks, uncomment the following line
\usepackage[cross,width=88truecm,height=123truecm,center]{crop}

\bibliography{bibliography.bib}

%%%%%%%%%%%%%%%%%%%%%%%%%%%%%%%%%%%%%%%%%%%%%%%%%%%%%%%%%%%%%%%%%%%%%%%%%%%%%%%%%%%%%%

\begin{document}

% We have a single poster frame.
\begin{frame}
  \begin{columns}[t]
    % ---------------------------------------------------------%
    % Set up a column
    \begin{column}{.45\textwidth}
      \begin{block}{Block 1}
        \lipsum[1-3]
      \end{block}

      \begin{block}{Block 2}
        \lipsum[3-4]
      \end{block}
    \end{column}
    % ---------------------------------------------------------%
    % end the column

    % ---------------------------------------------------------%
    % Set up a column 
    \begin{column}{.45\textwidth}
      \begin{block}{Yet another block}
        \lipsum[5-7]
      \end{block}

      \begin{block}{Some math block}
        \begin{equation}
          a^2+b^2=c^2
        \end{equation}
      \end{block}

      \begin{block}{Some citations}
        \alert{Watch out for } \cite{ff2010}
      \end{block}

      \begin{block}{References}
        \printbibliography
      \end{block}
    \end{column}
    % ---------------------------------------------------------%
    % end the column
  \end{columns}

  \begin{tikzpicture}[remember picture,overlay]
    \node[inner sep=0pt,xshift=-30cm,yshift=23cm] at (current page.east) {%
      \begin{postit}%
        Post-It time!%
      \end{postit}%
    }; 
  \end{tikzpicture}
  
\end{frame}

\end{document}
