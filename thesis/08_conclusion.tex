\chapter{Conclusion} \label{ch:conclusion}

Priority queues are one of the most important abstract data structures in computer science,
and much effort has been put into parallelizing them efficiently. In this paper,
we have outlined the evolution of concurrent priority queues from initial heap-based designs,
through a period of increasingly efficient SkipList queues, to current research into relaxed data structures.

The switch from heaps to SkipLists as the backing data structure highlights how a simple change in
direction can help revitalize an entire field of research. SkipList-based priority queues are the current
state of the art in strict shared-memory concurrent priority queues: they provide strong guarantees
and scale well to up to the tens of threads in practice. Important limiting factors are contention
at the front of the list and the large number of \ac{CAS} failures.
The \citeauthor{linden2013skiplist} queue is designed
to minimize the latter; but the former is inherent to all strict priority queues, which could mean
that the peak performance in such structures has been reached.

Recently invented relaxed priority queues do not exhibit the inherent bottleneck at the front of the list,
as they do not necessarily return the minimal element within the queue and are able to spread
\lstinline|DeleteMin| accesses over a larger area of the structure. In consequence, relaxed queues scale noticeably
better and to larger thread counts than strict designs. Further research is necessary in order to fully
explore the possibilities provided by relaxed data structures.
