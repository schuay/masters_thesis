\documentclass[a4paper,10pt]{article}

\usepackage{lmodern}
\usepackage[T1]{fontenc}
\usepackage[utf8]{inputenc}

\usepackage[backend=bibtex]{biblatex}
\usepackage{comment}
\usepackage{graphicx}
\usepackage[pdfborder={0 0 0}]{hyperref}
\usepackage{listings}
\usepackage[usenames,dvipsnames,table]{xcolor}

\bibliography{bibliography.bib}

\definecolor{Gray}{gray}{0.5}
\definecolor{OliveGreen}{cmyk}{0.64,0,0.95,0.40}

\pdfstringdefDisableCommands{\def\citeauthor#1{#1}}

\lstset{
    language=C++,
    basicstyle=\ttfamily,
    keywordstyle=\color{OliveGreen},
    commentstyle=\color{Gray},
    captionpos=b,
    breaklines=true,
    breakatwhitespace=false,
    showspaces=false,
    showtabs=false,
    numbers=left,
}

\title{Master's Thesis Proposal \\
       KLSM: A Relaxed Concurrent Priority Queue \\
       Technical University of Vienna}
\author{Jakob Gruber, 0203440 \\
        Advisor: Prof. Dr. Scient. Jesper Larsson Tr\"aff}

\begin{document}

\maketitle

\begin{comment}
http://www.informatik.tuwien.ac.at/dekanat/abschluss-master

Der Anmeldung der Diplomarbeit ist ein Abstract beizufügen. Das Abstract muss strukturiert in
i) Problemstellung,
ii) erwartetes Resultat,
iii) methodisches Vorgehen,
iv) State-of-the art (inkl. mind. vier Literaturreferenzen) sowie
v) Bezug zum angeführten Studium
abgefasst werden.

Bsp 1: http://www.informatik.tuwien.ac.at/dekanat/Abstract1.pdf
Bsp 2: http://www.informatik.tuwien.ac.at/dekanat/Abstract2.pdf
\end{comment}

\section{Motivation \& Problem Statement}

Priority queues are abstract data structures which store a set of key/value pairs
and allow efficient access to the item with the minimal (maximal) key. Such queues are an important
element in various areas of computer science such as algorithmics (i.e. Dijkstra's shortest
path algorithm) and operating system (i.e. priority schedulers).

The recent trend towards multiprocessor computing requires new implementations of basic
data structures which are able to be used concurrently and scale well to a large number
of threads. In particular, lock-free structures promise superior scalability by avoiding
the use of blocking synchronization primitives.

However, priority queues in particular are challenging to parallelize efficiently since
the \lstinline|delete_min| operation causes high contention at the minimal (maximal) element.
Even though concurrent priority queues have been extensively researched over the past decades,
a good solution has not yet been reached.

A recent promising approach has been through relaxation of provided guarantees, i.e.
allowing the priority queue to return one of the $k$ minimal items. The $k$-LSM is a lock-free
priority queue design which follows this approach and displays high scalability in initial
benchmarks. However, it is currently only available integrated into the task-scheduling framework
Pheet\footnote{\url{www.pheet.org}} and thus cannot be compared directly to other recent designs.

Within this thesis, a standalone version of the $k$-LSM priority queue will be developed and
compared extensively with state of the art concurrent priority queues.

\section{Expected Results}

\section{Methodology}

\section{State of the Art}

\section{Relevance to Software Engineering \& Internet Computing}

\nocite{*} % TODO: Remove me.
\printbibliography

\end{document}
